%%% template.tex
%%%
%%% This LaTeX source document can be used as the basis for your technical
%%% paper or abstract.

%%% The parameter to the ``documentclass'' command is very important.
%%% - use ``review'' for content submitted for review.
%%% - use ``preprint'' for accepted content you are making available.
%%% - use ``tog'' for technical papers accepted to the TOG journal and
%%%   for presentation at the SIGGRAPH or SIGGRAPH Asia conference.
%%% - use ``conference'' for final content accepted to a sponsored event
%%%   (hint: If you don't know, you should use ``conference.'')

\documentclass[conference]{acmsiggraph}

%%% Make the ``BibTeX'' word pretty...

\def\BibTeX{{\rm B\kern-.05em{\sc i\kern-.025em b}\kern-.08em
    T\kern-.1667em\lower.7ex\hbox{E}\kern-.125emX}}

\title{Neuroevolution in video games}

%%% TODO: Mettre les adresses de l'école
\author{Guillaume Ambrois\thanks{e-mail:guillaume.ambrois@gmail.com}, Sébastien Gaulier\thanks{e-mail:sebastien.gaulier@gmail.com}\\ESGI, Paris}
\pdfauthor{Guillaume Ambrois & Sébastien Gaulier}

%%% TODO: Ajouter des mots clés
\keywords{video game, artificial intelligence}

\begin{document}

%%% This is the ``teaser'' command, which puts an figure, centered, below 
%%% the title and author information, and above the body of the content.

%%% TODO: Changer l'image
 \teaser{
   \includegraphics[height=1.5in]{images/sampleteaser}
   \caption{Lucas}
 }

\maketitle

\begin{abstract}

In this sample paper, we describe the formatting requirements for
content accepted to SIGGRAPH-sponsored events. The same format can be
used for content ranging from a one-page Poster or Talk abstract, to a
full-length Technical Paper. Please review this document even if you
have submitted content to a SIGGRAPH-sponsored event in the past, as
some of the details have changed relative to previous years.

\end{abstract}

\keywordlist

%% Required for all content. 

\copyrightspace

\section{The Basics: General Organization}

CR categories and keywords are required for Technical Papers and
Technical Briefs, and optional for all other types of content. The
\LaTeX{} and Word templates faithfully implement the formatting
specifications in this document, and the sample PDF documents can be
used as visual examples for organization and layout.

\section{The Basics: Page Size and Columns}

Your content should be formatted on a US Letter (8.5 inches wide by 11
inches tall) page size. Please make sure you.re not working on an A4
page size. Page margins should be set at 0.75 inches on the right,
left, and top, and one inch at the bottom. Two columns should be used
throughout your content, except for the title, affiliations, and large
images or tables which span both columns.

\section{The Basics: Typefaces, Paragraphs, and Line Spacing}

Please use a serif (Times, Times New Roman, etc.) typeface for the
body of your content. This serif typeface should be used for
everything except the title of your content and section headings,
which should be set in a sans-serif (Helvetica, etc.) typeface.

Your content should be prepared with 9-point text and 10-point line
spacing. This includes the references section. Do not reduce the
typeface size or the line spacing of any part of your content, in an
effort to fit more into a specific number of pages. 

All typefaces used in your content must be embedded in the PDF you
create. 

Paragraphs are prepared without any indentation on the first line, and
with a 10-point-tall space between paragraphs.

Please do not add page numbers to your content. They will be added
during production.

\bibliographystyle{acmsiggraph}
\nocite{*}
\bibliography{template}
\end{document}
